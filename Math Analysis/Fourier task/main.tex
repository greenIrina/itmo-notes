\documentclass{article}

\usepackage[utf8]{inputenc} 
\usepackage[russian]{babel} 
\usepackage{amsmath} 
\usepackage{hyperref} 
\usepackage{graphicx}
\usepackage{amssymb}
\usepackage{wrapfig}
\usepackage{bbm}
\usepackage{parskip}
\usepackage{gensymb}
\usepackage{multicol}
\usepackage{array}
\usepackage[final]{pdfpages}
\parindent 0pt
\parskip 6pt
\newcommand{\numberset}[1]{\mathbb{#1}} 
\newcommand{\Aop}{\mathcal{A}}
\newcommand{\Bop}{\mathcal{B}}
\newcommand{\N}{\numberset{N}}
\newcommand{\Q}{\mathbb{Q}}
\newcommand{\R}{\mathbb{R}}
\newcommand{\Z}{\mathbb{Z}}
\newcommand{\Eps}{\mathcal{E}}
\newcommand{\Zero}{\mathbb{O}}
\newcommand{\Compl}{\mathbb{C}}
\newcommand{\n}{\bigbreak}
\usepackage[a4paper, total={6in, 8in}]{geometry}
\usepackage{color}
\usepackage{hyperref}
\def\letus{%
    \mathord{\setbox0=\hbox{$\exists$}%
             \hbox{\kern 0.125\wd0%
                   \vbox to \ht0{%
                      \hrule width 0.75\wd0%
                      \vfill%
                      \hrule width 0.75\wd0}%
                   \vrule height \ht0%
                   \kern 0.125\wd0}%
           }%
}
\hypersetup{
    colorlinks=true,
    linkcolor=blue,
    urlcolor=red,
    linktoc=all
}
\title{Матан Фурье}

\begin{document}
\maketitle 
Для функции $f (x)$, заданной на отрезке, построить три ряда Фурье: общий тригонометрический ряд, ряд Фурье по синусам и по косинусам. Для каждого полученного ряда построить (с помощью компьютера) графики нескольких частных сумм (например, $S_5,\,S_{10},\,S_{50}$) и график исходной функции. Убедиться (визуально), что частные суммы приближают исходную функцию. Знать, к чему сходятся построенные ряды Фурье в каждой точке.

$$f(x)=\begin{cases} \cos x,\,x\in \Big[ 0,\,\cfrac{\pi}{2}\,\Big) \\ 0,\,x\in \Big[\cfrac{\pi}{2},\,2\pi\Big] \end{cases}
$$
\tableofcontents
\newpage
\section{Общий тригонометрический ряд Фурье}
Наша функция задана на интервале $\Big[ 0,\,\cfrac{\pi}{2}\,\Big) \cup \Big[\cfrac{\pi}{2},\,2\pi\Big]$, это значит, что длина интервала $L = 2\pi$

Теперь мы можем посчитать коэффициенты Фурье по формулам:

$$ a_0 = \cfrac{1}{L}\cdot \int_{0}^{2\pi} f(x)dx = \cfrac{1}{L} \int_{0}^{\frac{\pi}{2}} \cos(x)dx=\cfrac{1}{2\pi}$$

$$ a_n=\cfrac{2}{L}\int_0^{2\pi} f(x)\cdot\cos\Bigg(\cfrac{2n\pi x}{L}\Bigg)dx= \cfrac{2}{L}\int_0^{\frac{\pi}{2}}\cos(x)\cdot\cos(nx)dx = -\cfrac{\cos\Big(\cfrac{\pi n}{2}\Big)}{\pi (n^2-1)}$$

$$ b_n=\cfrac{2}{L}\int_0^{2\pi} f(x)\cdot\sin\Bigg(\cfrac{2n\pi x}{L}\Bigg)dx= \cfrac{2}{L}\int_0^{\frac{\pi}{2}}\cos(x)\cdot\sin(nx)dx = \cfrac{n-\sin\Big(\cfrac{\pi n}{2}\Big)}{\pi (n^2-1)} $$

Как мы видим, определенные интегралы при $n = 1$ содержат части, при которых происходит деление на ноль, поэтому найдем значения при $n\to1$:
\begin{itemize}
    \item $L_1=\cfrac{1}{4}\cos(x)-a_n$ при $n=1$
    \item $L_2=\cfrac{1}{2\pi}\sin(x)-b_n$ при $n=1$
\end{itemize}
Итак, составим общий тригонометрический ряд Фурье по вычисленным коэффициентам по формуле:
$$ a_0+\sum_{n=1}^{\infty}\Bigg(a_n\cdot\cos\Bigg(\cfrac{2n\pi x}{L}\Bigg) + b_n\cdot\sin\Bigg(\cfrac{2n\pi x}{L}\Bigg)\Bigg) = $$

$$= \cfrac{1}{2\pi}+(L_1+L_2)+\Bigg(\sum_{n=2}^{\infty}(a_n\cdot\cos(nx)+b_n\cdot\sin(nx))\Bigg)  $$

\href{https://www.desmos.com/calculator/59wiprilp0}{Рисуночек}

\newpage
\section{Построить ряд Фурье по косинусам}

Посчитаем коэффициенты
$$ a_0 = \cfrac{2}{L}\cdot \int_{0}^{2\pi} f(x)dx = \cfrac{2}{L} \int_{0}^{\frac{\pi}{2}} \cos(x)dx=\cfrac{1}{\pi}$$

$$ a_n=\cfrac{2}{L}\int_0^{2\pi} f(x)\cdot\cos\Bigg(\cfrac{n\pi x}{L}\Bigg)dx= \cfrac{2}{L}\int_0^{\frac{\pi}{2}}\cos(x)\cdot\cos\Big(\cfrac{nx}{2}\Big)dx = -\cfrac{4\cos\Big(\cfrac{\pi n}{4}\Big)}{\pi (n^2-4)}$$

Составим ряд

$$ \cfrac{a_0}{2}+\sum_{n=1}^{\infty} a_n\cdot\cos\Bigg(\cfrac{n\pi x}{L}\Bigg) = \cfrac{1}{2\pi} + \sum_{n=1}^{\infty} a_n\cdot\cos\Big(\cfrac{nx}{2}\Big)$$

\href{https://www.desmos.com/calculator/bh6zh4bwxo}{Рисуночек}

\newpage
\section{Построить ряд Фурье по синусам}
Вычислим коэффициент
$$ b_n=\cfrac{2}{L}\int_0^{2\pi} f(x)\cdot\sin\Bigg(\cfrac{n\pi x}{L}\Bigg)dx= \cfrac{2}{L}\int_0^{\frac{\pi}{2}}\cos(x)\cdot\sin\Big(\cfrac{nx}{2}\Big)dx = \cfrac{2\Big(n-2\sin\Big(\cfrac{\pi n}{4}\Big)\Big)}{\pi(n^2-4)}$$

Составим ряд

$$ \sum=\sum_{n=1}^{\infty} b_n\cdot\sin\Big(\cfrac{nx}{2}\Big)$$

\href{https://www.desmos.com/calculator/anislzxvmj}{Рисуночек}

\end{document}